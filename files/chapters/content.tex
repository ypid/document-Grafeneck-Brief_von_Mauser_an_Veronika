zurzeit passiert so viel, was mich ablenkt. Kennst mich ja. Bei so etwas kann ich nicht anders, muss
der Sache nachgehen. Ich schreibe dir, um dir endlich alles zu sagen. Mir kommt es so vor als würd
ich dich betrügen, wenn ich es nicht tät. Also nun die ganze Geschichte.

Ich ging am Karsamstag zum Münzloch, die Höhl, die ich schon letztes Jahr entdeckt hatte. Ich kannte
mich aus, aber ich konnt erst dieses Jahr eine kleine Nebenhöhle erforschen. Die Lehmkammerhöhle
(der Name passt wirklich gut, ein klassisches Schlüssellochprofil aber mit Lehm ausgekleidet). Da
befand sich ein Zugang, der auf natürliche Weise mit Lehm versiegelt war. Nachdem ich die Kammer
geöffnet hatte und mich hineingezwängt hatte, sah ich eine mumifizierte Leiche. Die Leich war ein
Kerl, hatte ein Einschussloch beziehungsweise Austrittsloch im Auge und trug einen Anzug. Selbstmord
kann man wohl ausschließen, da keine Waffe in der Kammer zu finden war. Er wurde aber auch nicht
in der Höhl umgebracht (keine Patrone, kein Blut). Aber es ist mir unklar, wie der Körper hierher
kam. Es gibt keinen zweiten Eingang. Dieser Fund hat mich völlig gefesselt. Ich wusst, dass ich
diesen Fall allein lösen muss. Füllte mich zurückgeworfen in eine vergangene Zeit, die hier
konserviert auf mich wartete und mir Stück für Stück ihre Geheimnisse preisgeben würde. Zugleich
fühlte es sich so an, als ob mein Vater anwesend wäre. Ich nahm also die Ermittlungen auf.

Als ich dann bei dir war, konnte ich nichts erzählen und der Polizei erst recht nicht. Wie gesagt,
es war mein Fall, meine Vergangenheit. Obwohl ich damals wünschte, ich hätte nie diesen Fund
gemacht.
Aber auf der anderen Seite muss jede Geschichte ans Tageslicht.
Nachdem ich dann rausgekriegt hatte, dass die Leich aus der Zeit der Nazis stammen musste, kam alles
wieder hoch. Die Geschichte mit Mutz und meinem Vater, der ja damals Polizist war (ich hab dir die
Geschichten ja erzählt). Dieses ganze Unrecht, das damals passiert ist, die haben es alle ignoriert.
Aber mein Vater war anders, er hatte sich immer für das Recht eingesetzt und versucht, Juden zu
beschützen. Ich vermisse ihn~\dots

Da ich die Leich soweit es mir möglich war, untersucht hatte, bin ich zu Waiblinger gegangen und hab
ihm die Leich gemeldet (ein Osterei für ihn). Ich hätte es beinahe bereut, da er meinte ich hätte
den Fund sofort melden sollen. Er beharrte auch darauf, dass ich mein Kraftrad, wie er es nannte,
von der Straße entfernen sollte. Der in seinem Büro ist nicht der richtige für diesen Fall. Aber ich
machte meinen Bericht dann doch fertig. Wurde auch Zeit, dass unser Dorfpolizist mal was tut. Er
wollt die Leich auch direkt sehen, kam aber nicht durch die enge Öffnung der Höhl (der ist einfach
zu dick. Aber ich muss sagen, es ist wirklich nicht ganz leicht. Das letzte Stück zu der Kammer ist
ein enger Schluf, der auch noch einen Knick nach oben macht). Waiblinger hat stattdessen ein
Spezialkommando aus der Kreisstadt gerufen, um die Leich zu bergen. Ich hab in der Zeit die Proben,
die ich von der Leich genommen hatte, untersucht. Ich fand raus, dass er wohl oberhalb der Höhl
erschossen wurde (hab Waldboden unter den Fingernägeln gefunden). Als ich gerade dabei war, mir die
Gegend um das Münzloch anzusehen, kam der Waiblinger, hat das Gebiet auch noch abgesperrt und mich
weggeschickt. Hat wohl geahnt, was ich vorhatte, weißt. Aber von so was lasse ich mich nicht von
meinem Fall abbringen. Bin erstmal nach Haus gegangen und da kommt auch schon Kommissar Greving. Wir
gingen zur Lehmkammerhöhle. Aber auch Greving musste aufgeben, bevor er die Leich zu sehen bekam
(hat sich aber gut angestellt der Greving auch mit seiner Klaustrophobie). Er rief sein Team in
Reutlingen an.

Nach dem Ausflug mit Greving, machte ich mich vom Acker, kam aber mit meinem Metallsuchgerät wieder
her. Ich fand auch tatsächlich die Patronenhüls und das Projektil unter dem Waldboden. Doch als ich
die Patronenhüls mit denen meines Vaters verglich (du weiß ja, er hatte eine P 04),
durchfährt es mich. Es ist das gleiche Kaliber. 7,63. Als hätte Vater etwas mit der Sache zu tun.
Er, ein Mörder?

Als die Verstärkung von Greving da ist, musste ich schon wieder den Höhlenführer spielen. Die
Spurensicherung bekam es tatsächlich hin, die Leich aus der Lehmkammerhöhle zu transportieren. Die
haben da einen Plastiksack benutzt und die Leich dann durch die engen Gänge gezogen. Sie wird jetzt
in Tübingen untersucht. Greving ist mir sympathisch. Er ist ein guter Kommissar. Der will was
wissen. Auch wenn ich ihm nicht alles erzählen kann. Ich bin halt der Polizei einen Schritt voraus.

Bei dem Besuch des Judenfriedhofs mit dir musste ich dann einfach über meinen Vater reden. Es konnte
gut sein, dass er die Person in der Höhl ermordet hat. Diese Vorstellung hat mein Bild von ihm
total ins Wanken gebracht. Ich bekomme die alte Zeit nicht mehr aus meinem Gedächtnis. Muss ständig
daran denken.

Später hab ich dann mit Heinrich und Eugen geredet und erfahren, dass Greving jetzt die Leut im Dorf
ausfragt. Greving kam dann auch zu uns in den \enquote{Pflug}. Er meinte, dass auf dem Rücken der
Leich ein
Kreuz ist, mit Kreide gezeichnet. Heute weiß ich von Heinrich, dass es ein Zeichen für die gewesen
ist, die damals nach Grafeneck gebracht wurden, die Behinderten. Euthanasie nannten die das damals,
weißt.

Ich hab mich dann sogar nach Grabenstetten getraut, um mehr über die Höhle zu erfahren. War aber
leider nichts. Ach diese Technik, die wird nie die Indizes und Archive ersetzen können. Nach diesem
Misserfolg hab ich etwas getan, wovor ich mich längere Zeit schon gedrückt habe. Aber jetzt brauchte
ich einfach Gewissheit. Ich begann das Rillenprofil von dem Projektil, das ich bei der Höhl fand,
mit einem aus der Privatwaffe meines Vaters zu vergleichen. Ich weiß noch aus alten
Kriminalromanen, dass jede Waffe beim Abschießen ein individuelles Muster in das Projektil treibt.
Und in der Schule standen mir die geeigneten Instrumente zur Verfügung. Nachdem ich mit dieser
langwierigen Aufgabe fertig war, gab es keinen Zweifel mehr. Die Waffe meines Vaters war die
Mordwaffe. Hat sich also bestätigt, was ich befürchtet hatte. Es besteht natürlich immer noch die
Möglichkeit, dass mein Vater nicht selbst abgedrückt hat, aber dass werde ich womöglich nie
erfahren. Auch der Todeszeitpunkt der Leiche passt (vor 50 Jahren). Auch wenn der Todeszeitpunkt
ausgehend von der Leich nicht genau ermittelt werden kann, da sie ja mumifiziert ist. Er ist aber
wohl zur Zeit der Nazis ermordet worden, meinte Greving. Das wusst ich vor ihm aber auch schon.

Nachdem ich das alles wusst, beschloss ich zu dem Arzt Hochstetter (er war damals für die
Todesurteile der Behinderten zuständig) zu fahren. Dieses Schwein Hochstetter war mit verantwortlich
für all die Morde und lebt nun, als wäre nichts gewesen in Hundersingen. Ich fuhr aber nicht nur
wegen der Leich, auch wegen meiner Schwester Mutz und der Ungerechtigkeit, die für mich immer noch
greifbar ist, hin. Ich nahm sogar die Waff von Vater mit. Die ganze Wut, die sich in mir aufgestaut
hatte, ließ ich bei ihm aus. Ich zielte mit der Pistol auf ihn. Hochstetter sagte, dass ich ihn in
Ruhe lassen sollt und lieber mit Eugen sprechen solle.

Das tat ich dann auch. Er gab zu, dass er damals die grauen Busse gefahren hatte, die die
Behinderten
abgeliefert haben und dass er vor allen anderen gewusst hat, was in Grafeneck vor sich ging. Und er
hat all die Jahre geschwiegen, hat mit niemandem gesprochen! Er will keine Schuld auf sich
nehmen! Das hat mich in den Wahnsinn getrieben. Ich holte wieder Vaters Waff raus und hielt sie
Eugen direkt an die Stirn. Hab aber nicht abgedrückt. Ich kann das nicht, kann keinen Menschen
umbringen. Wollt einfach, dass irgendwer das Schweigen bricht und darüber redet. Er sollt es einfach
zugeben. Ist dann aber wütend abgehauen.

Ich hätt sie fast umgebracht~\dots

Ich muss aber jetzt endlich mit jemandem darüber reden. Anders halt ich das nicht aus.
Ich war deswegen auch kurz davor, zu dir in die Töpferwerkstatt zu kommen und dir alles bei einer
Tasse Kaffe zu erzählen. Aber ich war einfach zu aufgebracht. Ich denk ich kann in einem Brief viel
besser und überlegter sagen, was mich beschäftigt.

Du warst damals einfach nicht dabei. Aber ich hoffe trotzdem, dass du mich verstehen kannst. Falls
du nun diesen Brief in Händen hältst, dann konnte ich mich dazu durchringen ihn dir in den
Briefkasten zu werfen~\dots
